\documentclass[12pt]{article}

\usepackage{fullpage}
\usepackage{multicol,multirow}
\usepackage{tabularx}
\usepackage{ulem}
\usepackage[utf8]{inputenc}
\usepackage[russian]{babel}
\usepackage{minted}

\usepackage{color} %% это для отображения цвета в коде
\usepackage{listings} %% собственно, это и есть пакет listings

\lstset{ %
language=C++,                 % выбор языка для подсветки (здесь это С++)
basic style=\small\sffamily, % размер и начертание шрифта для подсветки кода
numbers=left,               % где поставить нумерацию строк (слева\справа)
%numberstyle=\tiny,           % размер шрифта для номеров строк
step number=1,                   % размер шага между двумя номерами строк
numbersep=5pt,                % как далеко отстоят номера строк от подсвечиваемого кода
background color=\color{white}, % цвет фона подсветки - используем \usepackage{color}
show spaces=false,            % показывать или нет пробелы специальными отступами
showstringspaces=false,      % показывать или нет пробелы в строках
show tabs=false,             % показывать или нет табуляцию в строках
frame=single,              % рисовать рамку вокруг кода
tab size=2,                 % размер табуляции по умолчанию равен 2 пробелам
captions=t,              % позиция заголовка вверху [t] или внизу [b]
break lines=true,           % автоматически переносить строки (да\нет)
breakatwhitespace=false, % переносить строки только если есть пробел
escape inside={\%*}{*)}   % если нужно добавить комментарии в коде
}


\begin{document}
\begin{titlepage}
\begin{center}
\textbf{МИНИСТЕРСТВО ОБРАЗОВАНИЯ И НАУКИ РОССИЙСКОЙ ФЕДЕРАЦИИ
\medskip
МОСКОВСКИЙ АВИАЦИОННЫЙ ИНСТИТУТ
(НАЦИОНАЛЬНЫЙ ИССЛЕДОВАТЕЛЬСКИЙ УНИВЕРСТИТЕТ)
\vfill\vfill
{\Huge ЛАБОРАТОРНАЯ РАБОТА №3} \\
по курсу объектно-ориентированное программирование
I семестр, 2021/22 уч. год}
\end{center}
\vfill

Студент \uline{\it {Абаровский Олег Александрович, группа М8О-207Б-20}\hfill}

Преподаватель \uline{\it {Дорохов Евгений Павлович}\hfill}

\vfill
\end{titlepage}

\subsection*{Условие}

Задание: \
Вариант 1: Треугольник, Квадрат, Прямоугольник.\
Необходимо спроектировать и запрограммировать на языке C++ классы трех фигур, согласно варианту задания. Классы должны удовлетворять следующим правилам:
\begin{enumerate}
\item Должны быть названы также, как в вариантах задания и расположенны в раздельных файлах: отдельно заголовки (имя\_класса\_с\_маленькой\_буквы.h), отдельно описание методов (имя\_класса\_с\_маленькой\_буквы.cpp).
\item Иметь общий родительский класс Figure;
\item Содержать конструктор, принимающий координаты вершин фигуры из стандартного потока std::cin, расположенных через пробел. Пример: "0.0 0.0 1.0 0.0 1.0 1.0 0.0 1.0"
\item Содержать набор общих методов:
\begin{itemize}
    \item size\_t VertexesNumber() - метод, возвращающий количество вершин фигуры;
    \item double Area() - метод расчета площади фигуры;
    \item void Print(std::ostream& os) - метод печати типа фигуры и ее координат вершин в поток вывода os в формате: "Rectangle: (0.0, 0.0) (1.0, 0.0) (1.0, 1.0) (0.0, 1.0)" с переводом строки в конце.
\end{itemize}
\end{enumerate}

\subsection*{Описание программы}

Исходный код лежит в 11 файлах:
\begin{enumerate}
\item src/main.cpp: основная программа, взаимодействие с пользователем посредством комманд из меню

\item include/figure.h:    описание абстрактного класса фигур

\item include/point.h:     описание класса точки
\item include/triangle.h:  описание класса треугольника, наследующегося от figures
\item include/rectangle.h: описание класса прямоугольника, наследующегося от figures
\item include/square.h:    описание класса квадрата, наследующегося от rectangle

\item include/point.cpp:     реализация класса точки
\item include/triangle.cpp:  реализация класса треугольника, наследующегося от figures
\item include/rectangle.cpp: реализация класса прямоугольника, наследующегося от figures
\item include/square.cpp:    реализация класса квадрата, наследующегося от rectangle

\end{enumerate}

\subsection*{Дневник отладки}
12.11 Ошибка: неправильно реализован метод печати типа фигуры и координат её вершин в поток вывода os.
\newline Решение: исправить формат вывода координат фигуры.

\subsection*{Недочёты}
Недочётов нет

\subsection*{Выводы}
Работа была полезной, в ходе её выполнения я получил представление об основных принципах объектно-ориентированного программирования. В моём варианте можно не только наследовать классы каждой фигуры от класса figures, но и унаследовать класс одной фигуры от класса другой(квадрат наследуется из прямоугольника)! Это хорошо иллюстрирует принцип наследуемости. Таким образом, эта работа оказалась для меня отличным введением в объектно-ориентированное программирование.


\vfill

\subsection*{Исходный код}

{\Huge figure.h}
\begin{minted}{c++}
#ifndef FIGURE_H
#define FIGURE_H

#include <iostream>
#include "point.h"

class Figure {
public:
  virtual size_t VertexesNumber() = 0;
  virtual double Area() = 0;
  virtual void Print(std::ostream &os) = 0;

};

#endif // FIGURE_H

\end{minted}
    \pagebreak

\begin{flushleft}
{\Huge point.h}
\begin{minted}{c++}
#ifndef POINT_H
#define POINT_H

#include <iostream>

class Point {
public:
  Point();
  Point(std::istream &is);
  Point(double x, double y);

  double dist(Point& other);
  
  friend std::istream& operator>>(std::istream& is, Point& p);
  friend std::ostream& operator<<(std::ostream& os, Point& p);
  Point &operator=(const Point &p);

private:
  double x_;
  double y_;
};

#endif // POINT_H

\end{minted}
\end{flushleft}
    \pagebreak

\begin{flushleft}
{\Huge point.cpp}
\begin{minted}{c++}
#include "point.h"

#include <cmath>

Point::Point() : x_(0.0), y_(0.0) {}

Point::Point(double x, double y) : x_(x), y_(y) {}

Point::Point(std::istream &is) {
  is >> x_ >> y_;
}

double Point::dist(Point& other) {
  double dx = (other.x_ - x_);
  double dy = (other.y_ - y_);
  return std::sqrt(dx*dx + dy*dy);
}

std::istream& operator>>(std::istream& is, Point& p) {
  is >> p.x_ >> p.y_;
  return is;
}

std::ostream& operator<<(std::ostream& os, Point& p) {
  os << "(" << p.x_ << ", " << p.y_ << ")";
  return os;
}

Point &Point::operator=(const Point &p) {
  this->x_ = p.x_;
  this->y_ = p.y_;
  return *this;
}
\end{minted}
\end{flushleft}
    \pagebreak

\begin{flushleft}
{\Huge main.cpp}
\begin{minted}{c++}
#include "figure.h"
#include "triangle.h"
#include "square.h"
#include "rectangle.h"

int main()
{
    std::cout << "Enter triangle:\n";
    Triangle trngl(std::cin);
    trngl.Print(std::cout);
    std::cout << "The number of vertexes: " << trngl.VertexesNumber() << "\n";
    std::cout << "Area: " << trngl.Area() << "\n\n";

    std::cout << "Enter square:\n";
    Square sqr(std::cin);
    sqr.Print(std::cout);
    std::cout << "The number of vertexes: " << sqr.VertexesNumber() << "\n";
    std::cout << "Area: " << sqr.Area() << "\n\n";

    std::cout << "Enter rectangle:\n";
    Rectangle rect(std::cin);
    rect.Print(std::cout);
    std::cout << "The number of vertexes: " << rect.VertexesNumber() << "\n";
    std::cout << "Area: " << rect.Area() << "\n";
    return 0;
}
\end{minted}
\end{flushleft}
    \pagebreak

\begin{flushleft}
{\Huge triangle.h}
\begin{minted}{c++}
#ifndef TRIANGLE_H
#define TRIANGLE_H
#include "figure.h"

class Triangle : public Figure {
 private:
  Point a_, b_, c_;
 public:
  Triangle();
  Triangle(const Triangle &triangle);
  Triangle(std::istream &is);
  size_t VertexesNumber();
  double Area();
  void Print(std::ostream &os);

};

#endif //TRIANGLE_H
\end{minted}
\end{flushleft}
    \pagebreak

\begin{flushleft}
{\Huge triangle.cpp}
\begin{minted}{c++}
#include "triangle.h"
#include <math.h>

Triangle::Triangle() : a_(0, 0), b_(0, 0), c_(0, 0) {}

Triangle::Triangle(const Triangle &triangle) {
  this->a_ = triangle.a_;
  this->b_ = triangle.b_;
  this->c_ = triangle.c_;
}

Triangle::Triangle(std::istream &is) {
  std::cin >> a_ >> b_ >> c_;
}

size_t Triangle::VertexesNumber() {
  return 3;
}

double Triangle::Area() {
  double a = a_.dist(b_);
  double b = b_.dist(c_);
  double c = c_.dist(a_);
  double p = (a + b + c) / 2;
  return sqrt(p * (p - a) * (p - b) * (p - c));
}

void Triangle::Print(std::ostream &os) {
  std::cout << "Triangle " << a_ << b_ << c_ << std::endl;
}
\end{minted}
\end{flushleft}
    \pagebreak

\begin{flushleft}
{\Huge rectangle.h}
\begin{minted}{c++}
#ifndef RECTANGLE_H
#define RECTANGLE_H

#include "figure.h"

class Rectangle : public Figure {
 private:
  Point a_, b_, c_, d_;

 public:
  Rectangle();
  Rectangle(const Rectangle &rectangle);
  Rectangle(std::istream &is);
  size_t VertexesNumber();
  double Area();
  void Print(std::ostream &os);

};

#endif //RECTANGLE_H
\end{minted}
\end{flushleft}
    \pagebreak

\begin{flushleft}
{\Huge rectangle.cpp}
\begin{minted}{c++}
#include "rectangle.h"

Rectangle::Rectangle() : a_(0, 0), b_(0, 0), c_(0, 0), d_(0, 0) {}

Rectangle::Rectangle(const Rectangle &rectangle) {
  this->a_ = rectangle.a_;
  this->b_ = rectangle.b_;
  this->c_ = rectangle.c_;
  this->d_ = rectangle.d_;
}

Rectangle::Rectangle(std::istream &is) {
  std::cin >> a_ >> b_ >> c_ >> d_;
}

size_t Rectangle::VertexesNumber() {
  return 4;
}

double Rectangle::Area() {
  double a = a_.dist(b_);
  double b = b_.dist(c_);
  return a * b;
}

void Rectangle::Print(std::ostream &os) {
  std::cout << "Rectangle " << a_ << b_ << c_ << d_ << std::endl;
}
\end{minted}
\end{flushleft}
    \pagebreak

\begin{flushleft}
{\Huge square.h}
\begin{minted}{c++}
#ifndef SQUARE_H
#define SQUARE_H

#include "figure.h"

class Square : public Figure {
 private:
  Point a_, b_, c_, d_;
 public:
  Square();
  Square(const Square &square);
  Square(std::istream &is);
  size_t VertexesNumber();
  double Area();
  void Print(std::ostream &os);
};

#endif //SQUARE_H
\end{minted}
\end{flushleft}
    \pagebreak

\begin{flushleft}
{\Huge square.cpp}
\begin{minted}{c++}

#include "square.h"

Square::Square() : a_(0, 0), b_(0, 0), c_(0, 0), d_(0, 0) {}

Square::Square(const Square &square) {
  this->a_ = square.a_;
  this->b_ = square.b_;
  this->c_ = square.c_;
  this->d_ = square.d_;
}

Square::Square(std::istream &is) {
  std::cin >> a_ >> b_ >> c_ >> d_;
}

size_t Square::VertexesNumber() {
  return (size_t)4;
}

double Square::Area() {
  double a = a_.dist(b_);
  return a * a;
}

void Square::Print(std::ostream &os) {
  std::cout << "Square " << a_ << b_ << c_ << d_ << std::endl;
}
\end{minted}
\end{flushleft}

\end{document}
