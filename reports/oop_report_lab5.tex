\documentclass[12pt]{article}

\usepackage{fullpage}
\usepackage{multicol,multirow}
\usepackage{tabularx}
\usepackage{ulem}
\usepackage[utf8]{inputenc}
\usepackage[russian]{babel}
\usepackage{minted}

\usepackage{color} %% это для отображения цвета в коде
\usepackage{listings} %% собственно, это и есть пакет listings

\lstset{ %
language=C++,                 % выбор языка для подсветки (здесь это С++)
basic style=\small\sffamily, % размер и начертание шрифта для подсветки кода
numbers=left,               % где поставить нумерацию строк (слева\справа)
%numberstyle=\tiny,           % размер шрифта для номеров строк
step number=1,                   % размер шага между двумя номерами строк
numbersep=5pt,                % как далеко отстоят номера строк от подсвечиваемого кода
background color=\color{white}, % цвет фона подсветки - используем \usepackage{color}
show spaces=false,            % показывать или нет пробелы специальными отступами
showstringspaces=false,      % показывать или нет пробелы в строках
show tabs=false,             % показывать или нет табуляцию в строках
frame=single,              % рисовать рамку вокруг кода
tab size=2,                 % размер табуляции по умолчанию равен 2 пробелам
captions=t,              % позиция заголовка вверху [t] или внизу [b]
break lines=true,           % автоматически переносить строки (да\нет)
breakatwhitespace=false, % переносить строки только если есть пробел
escape inside={\%*}{*)}   % если нужно добавить комментарии в коде
}


\begin{document}
\begin{titlepage}
\begin{center}
\textbf{МИНИСТЕРСТВО ОБРАЗОВАНИЯ И НАУКИ РОССИЙСКОЙ ФЕДЕРАЦИИ
\medskip
МОСКОВСКИЙ АВИАЦИОННЫЙ ИНСТИТУТ
(НАЦИОНАЛЬНЫЙ ИССЛЕДОВАТЕЛЬСКИЙ УНИВЕРСТИТЕТ)
\vfill\vfill
{\Huge ЛАБОРАТОРНАЯ РАБОТА №5} \\
по курсу объектно-ориентированное программирование
I семестр, 2021/22 уч. год}
\end{center}
\vfill

Студент \uline{\it {Абаровский Олег Александрович, группа М8О-207Б-20}\hfill}

Преподаватель \uline{\it {Дорохов Евгений Павлович}\hfill}

\vfill
\end{titlepage}

\subsection*{Условие}

Задание: \
Вариант 1: Динамический массив.\
Необходимо спроектировать и запрограммировать на языке C++ класс-контейнер первого уровня, содержащий одну фигуру (колонка фигура 1), согласно вариантам задания. Классы должны удовлетворять следующим правилам:
\begin{enumerate}
\item Требования к классу фигуры аналогичны требованиям из лабораторной работы 1.
\item Требования к классу контейнера аналогичны требованиям из лабораторной работы №2;
\item Класс-контейнер должен содержать объекты используя std::shared\_ptr<…>.
\end{enumerate}

\subsection*{Описание программы}

Исходный код лежит в 8 файлах:
\begin{enumerate}
\item src/main.cpp: основная программа, взаимодействие с пользователем посредством комманд из меню

\item include/figure.h:    описание абстрактного класса фигур

\item include/point.h:     описание класса точки
\item include/triangle.h:  описание класса треугольника, наследующегося от figures
\item include/tvector.h:     описание класса контейнера - динамического массива, содержащего объект с помощью умного указателя

\item include/point.cpp:     реализация класса точки
\item include/triangle.cpp:  реализация класса треугольника, наследующегося от figures
\item include/tvector.cpp: реализация класса контейнера - динамического массива, содержащего объект с помощью умного указателя

\end{enumerate}

\subsection*{Дневник отладки}
Исправлений не потребовалось.


\subsection*{Недочёты}


\subsection*{Выводы}
В ходе выполнения работы я познакомился с умными указателями, их задачами и преимуществами по сравнению с обычными указателями, такими как освобождение памяти при выходе объекта класса из области видимости и перемещение памяти. Изучил реализацию умных указателей в языке C++ - shared\_ptr. В итоге удалось создать контейнер, содержащий объекты, используя shared\_ptr.


\vfill

\subsection*{Исходный код}

{\Huge figure.h}
\begin{minted}{c++}
#ifndef FIGURE_H
#define FIGURE_H

#include <iostream>
#include "point.h"

class Figure {
public:
  virtual size_t VertexesNumber() = 0;
  virtual double Area() = 0;
  virtual void Print(std::ostream &os) = 0;

};

#endif // FIGURE_H

\end{minted}
    \pagebreak

\begin{flushleft}
{\Huge point.h}
\begin{minted}{c++}
#ifndef POINT_H
#define POINT_H

#include <iostream>

class Point {
public:
  Point();
  Point(std::istream &is);
  Point(double x, double y);

  double dist(Point& other);
  
  friend std::istream& operator>>(std::istream& is, Point& p);
  friend std::ostream& operator<<(std::ostream& os, Point& p);
  Point &operator=(const Point &p);

private:
  double x_;
  double y_;
};

#endif // POINT_H

\end{minted}
\end{flushleft}
    \pagebreak

\begin{flushleft}
{\Huge point.cpp}
\begin{minted}{c++}
#include "point.h"

#include <cmath>

Point::Point() : x_(0.0), y_(0.0) {}

Point::Point(double x, double y) : x_(x), y_(y) {}

Point::Point(std::istream &is) {
  is >> x_ >> y_;
}

double Point::dist(Point& other) {
  double dx = (other.x_ - x_);
  double dy = (other.y_ - y_);
  return std::sqrt(dx*dx + dy*dy);
}

std::istream& operator>>(std::istream& is, Point& p) {
  is >> p.x_ >> p.y_;
  return is;
}

std::ostream& operator<<(std::ostream& os, Point& p) {
  os << "(" << p.x_ << ", " << p.y_ << ")";
  return os;
}

Point &Point::operator=(const Point &p) {
  this->x_ = p.x_;
  this->y_ = p.y_;
  return *this;
}
\end{minted}
\end{flushleft}
    \pagebreak

\begin{flushleft}
{\Huge main.cpp}
\begin{minted}{c++}
#include <iostream>
#include "triangle.h"
#include "tvector.h"

using namespace std;

int main()
{
	cout << "Comands:" << endl;
	cout << "a - add new triangle (a [input])" << endl;
	cout << "d - erase triangle by index (d [idx])" << endl;
	cout << "s - set triangle by index (s [idx] [input])" << endl;
	cout << "p - print all containing triangles (p)" << endl;
	cout << "q - quit (q)" << endl;
	char f = 1;
	TVector *vect = new TVector();
	char cmd;
	while(f)
	{
		cout << "> ";
		cin >> cmd;
		switch(cmd)
		{
			case 'a':
			{
				vect->InsertLast(Triangle(cin));
				break;
			}
			case 'd':
			{
				int di;
				cin >> di;
				vect->Erase(di);
				break;
			}
			case 's':
			{
				int si;
				cin >> si;
				Triangle csq(cin);
				(*vect)[si] = csq;
				break;
			}
			case 'p':
			{
				cout << *vect << endl;
				break;
			}
			case 'q':
			{
				f = 0;
				break;
			}
			default:
				cout << "wrong input" << endl;
		}
	}
	delete vect;
}
\end{minted}
\end{flushleft}
    \pagebreak

\begin{flushleft}
{\Huge triangle.h}
\begin{minted}{c++}
#ifndef TRIANGLE_H
#define TRIANGLE_H
#include "figure.h"

class Triangle : public Figure {
 private:
  Point a_, b_, c_;
 public:
  Triangle();
  Triangle(const Triangle &triangle);
  Triangle(std::istream &is);
  size_t VertexesNumber();
  double Area();
  void Print(std::ostream &os);

};

#endif //TRIANGLE_H
\end{minted}
\end{flushleft}
    \pagebreak

\begin{flushleft}
{\Huge triangle.cpp}
\begin{minted}{c++}
#include "triangle.h"
#include <math.h>

Triangle::Triangle() : a_(0, 0), b_(0, 0), c_(0, 0) {}

Triangle::Triangle(const Triangle &triangle) {
  this->a_ = triangle.a_;
  this->b_ = triangle.b_;
  this->c_ = triangle.c_;
}

Triangle::Triangle(std::istream &is) {
  std::cin >> a_ >> b_ >> c_;
}

size_t Triangle::VertexesNumber() {
  return 3;
}

double Triangle::Area() {
  double a = a_.dist(b_);
  double b = b_.dist(c_);
  double c = c_.dist(a_);
  double p = (a + b + c) / 2;
  return sqrt(p * (p - a) * (p - b) * (p - c));
}

void Triangle::Print(std::ostream &os) {
  std::cout << "Triangle " << a_ << b_ << c_ << std::endl;
}
\end{minted}
\end{flushleft}
    \pagebreak

\begin{flushleft}
{\Huge tvector.h}
\begin{minted}{c++}
//TVECTOR.H
#ifndef TVECTOR_H
#define TVECTOR_H
#define et_tvector std::shared_ptr<Figure>

#include <iostream>
#include "triangle.h"
#include <memory>

class TVector
{
	public:
		TVector();
		TVector(const TVector& other);
		void Erase(int pos);
		void InsertLast(const et_tvector& elem);
		void RemoveLast();
		const et_tvector& Last();
		et_tvector& operator[](const size_t idx);
		bool Empty();
		size_t Length();
		friend std::ostream& operator<<(std::ostream& os, TVector& obj);
		void Clear();
		~TVector();
	private:
		void resize(int newsize);
		et_tvector *vals;
		int len;
		int rLen;
};

#endif
\end{minted}
\end{flushleft}
    \pagebreak

\begin{flushleft}
{\Huge tvector.cpp}
\begin{minted}{c++}
//TVECTOR.CPP
#include "tvector.h"
#include <iostream>
#include <cstring>

TVector::TVector()
{
	vals = NULL;
	len = 0;
	rLen = 0;
}

TVector::TVector(const TVector& other)
{
	len = other.len;
	rLen = other.rLen;
	vals = (et_tvector*)malloc(sizeof(et_tvector)*len);
	memcpy((void*)vals, (void*)other.vals, sizeof(et_tvector)*len);
}

void TVector::Erase(int pos)
{
	if(len == 1)
	{
		Clear();
		return;
	}
	vals[pos] = NULL;
	memmove((void*)&(vals[pos]),(void*)&(vals[pos+1]),sizeof(et_tvector)*(len-pos-1));
	len--;
	if(len==rLen>>1)
		resize(len);
}

void TVector::InsertLast(const et_tvector& elem)
{
	if(rLen)
	{
		if(len>=rLen)
		{
			rLen<<=1;
			resize(rLen);
		}
	}
	else
	{
		rLen=1;
		resize(rLen);
	}
	vals[len] = elem;
	len++;
}

void TVector::RemoveLast()
{
	Erase(len-1);
}

const et_tvector& TVector::Last()
{
	return vals[len-1];
}

et_tvector& TVector::operator[](const size_t idx)
{
	return vals[idx];
}

bool TVector::Empty()
{
	return len == 0;
}

size_t TVector::Length()
{
	return len;
}

std::ostream& operator<<(std::ostream& os, TVector& obj)
{
	os << '[';
	for(int i = 0; i < obj.len; i++)
	{
		os << obj.vals[i].get()->Area();
		if(i != obj.len - 1)
			os << " ";
	}
	os << ']';
	return os;
}

void TVector::Clear()
{
	if(!Empty())
	{
		for(int i=0;i<len;i++)
			vals[i]=NULL;
		free(vals);
		vals = NULL;
		len = 0;
		rLen = 0;
	}
}

void TVector::resize(int newsize)
{
	vals = (et_tvector*)realloc((void*)vals, sizeof(et_tvector)*newsize);
}

TVector::~TVector()
{
	Clear();
}
\end{minted}
\end{flushleft}

\end{document}
