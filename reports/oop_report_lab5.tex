\documentclass[12pt]{article}

\usepackage{fullpage}
\usepackage{multicol,multirow}
\usepackage{tabularx}
\usepackage{ulem}
\usepackage[utf8]{inputenc}
\usepackage[russian]{babel}
\usepackage{minted}

\usepackage{color} %% это для отображения цвета в коде
\usepackage{listings} %% собственно, это и есть пакет listings

\lstset{ %
language=C++,                 % выбор языка для подсветки (здесь это С++)
basic style=\small\sffamily, % размер и начертание шрифта для подсветки кода
numbers=left,               % где поставить нумерацию строк (слева\справа)
%numberstyle=\tiny,           % размер шрифта для номеров строк
step number=1,                   % размер шага между двумя номерами строк
numbersep=5pt,                % как далеко отстоят номера строк от подсвечиваемого кода
background color=\color{white}, % цвет фона подсветки - используем \usepackage{color}
show spaces=false,            % показывать или нет пробелы специальными отступами
showstringspaces=false,      % показывать или нет пробелы в строках
show tabs=false,             % показывать или нет табуляцию в строках
frame=single,              % рисовать рамку вокруг кода
tab size=2,                 % размер табуляции по умолчанию равен 2 пробелам
captions=t,              % позиция заголовка вверху [t] или внизу [b]
break lines=true,           % автоматически переносить строки (да\нет)
breakatwhitespace=false, % переносить строки только если есть пробел
escape inside={\%*}{*)}   % если нужно добавить комментарии в коде
}


\begin{document}
\begin{titlepage}
\begin{center}
\textbf{МИНИСТЕРСТВО ОБРАЗОВАНИЯ И НАУКИ РОССИЙСКОЙ ФЕДЕРАЦИИ
\medskip
МОСКОВСКИЙ АВИАЦИОННЫЙ ИНСТИТУТ
(НАЦИОНАЛЬНЫЙ ИССЛЕДОВАТЕЛЬСКИЙ УНИВЕРСТИТЕТ)
\vfill\vfill
{\Huge ЛАБОРАТОРНАЯ РАБОТА №1} \\
по курсу объектно-ориентированное программирование
I семестр, 2021/22 уч. год}
\end{center}
\vfill

Студент \uline{\it {Абаровский Олег Александрович, группа М8О-207Б-20}\hfill}

Преподаватель \uline{\it {Дорохов Евгений Павлович}\hfill}

\vfill
\end{titlepage}

\subsection*{Условие}

Разработать программу на языке C++ согласно варианту задания. Программа на C++
должна собираться с помощью системы сборки CMake. Программа должна получать
данные из стандартного ввода и выводить данные в стандартный вывод. \
Необходимо настроить сборку лабораторной работы с помощью CMake. Собранная
программа должна называться oop_exercise_01 \
Задание: \
Вариант 1: Комплексное число в алгебраической форме представляются парой
действительных чисел (a, b), где a – действительная часть, b – мнимая
часть. Реализовать класс Complex для работы с комплексными числами.
Обязательно должны быть присутствовать операции
- сложения add, (a, b) + (c, d) = (a + c, b + d);
- вычитания sub, (a, b) – (c, d) = (a – c, b – d);
- умножения mul, (a, b) ´ (c, d) = (ac – bd, ad + bc);
- деления div, (a, b) / (c, d) = (ac + bd, bc – ad) / (c 2 + d 2 );
- сравнение equ, (a, b) = (c, d), если (a = c) и (b = d);
- сопряженное число conj, conj(a, b) = (a, –b).

\subsection*{Описание программы}

Исходный код лежит в файле main.cpp, в котором реализован класс комплексного числа и описаны методы этого класса, среди которых сложение, вычитание, умножение, деление, сравнение, нахождение сопряженного числа.

\subsection*{Дневник отладки}
Исправлений не потребовалось.


\subsection*{Недочёты}


\subsection*{Выводы}
В ходе выполнения лабораторной работы я изучил реализацию классов в c++ и создал класс с набором методов согласно варианту задания. Работа была полезной, так как в результате я  получил представление о таком принципе объектно-ориентированного программирования, как инкапсуляция.


\vfill

\subsection*{Исходный код}

{\Huge main.cpp}
\begin{minted}{c++}
#include <iostream> 
#include <stdio.h>
using namespace std;
class Complex 
{
public:  
	double re, im; 
    Complex(){}
    Complex(int x, int y)
    {
        re = x;
        im = y;
    }
	Complex add(Complex a)
	{
	   	double resre, resim;
	   	resre = a.re + re;
	   	resim = a.im + im;
	   	return Complex(resre, resim);
	}
	Complex sub(Complex a)
	{
	   	double resre, resim;
	   	resre = re - a.re;
	   	resim = im - a.im;
	   	return Complex(resre, resim);
	}
	Complex mul(Complex a)
	{
	   	double resre, resim;
	   	resre = a.re * re - a.im * im;
	   	resim = re * a.im + im * a.re;
	   	return Complex(resre, resim);
	}
	Complex div(Complex a)
	{
	   	double resre, resim;
	   	resre = (re * a.re + im * a.im) / (a.re*a.re + a.im*a.im);
	   	resim = (a.re * im - re * a.im) / (a.re*a.re + a.im*a.im);
	   	return Complex(resre, resim);
	}
	bool equ(Complex a)
	{
	   	return (re == a.re && im == a.im);
	}
	Complex conj()
	{
	   	return Complex(re, -im);
	}
	~Complex(){}
};

int main()
{
	double re, im;
	bool f;
	Complex com1, com2, com;
	printf("Enter complex number one\n");
	scanf("%lf %lf", &re, &im);
	com1 = Complex(re, im);
	printf("Enter complex number two\n");
	scanf("%lf %lf", &re, &im);
	com2 = Complex(re, im);
	com = com1.add(com2);
	printf("Sum = (%lf, %lf)\n", com.re, com.im);
	com = com1.sub(com2);
	printf("Sub = (%lf, %lf)\n", com.re, com.im);
	com = com1.mul(com2);
	printf("Mul = (%lf, %lf)\n", com.re, com.im);
	com = com1.div(com2);
	printf("Div = (%lf, %lf)\n", com.re, com.im);
	f = com1.equ(com2);
	printf("Equ = %d\n", f);
	com = com1.conj();
	printf("Conj1 = (%lf, %lf)\n", com.re, com.im);
	com = com2.conj();
	printf("Conj2 = (%lf, %lf)\n", com.re, com.im);
	
}
\end{minted}
    \pagebreak


\end{document}
