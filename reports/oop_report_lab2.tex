\documentclass[12pt]{article}

\usepackage{fullpage}
\usepackage{multicol,multirow}
\usepackage{tabularx}
\usepackage{ulem}
\usepackage[utf8]{inputenc}
\usepackage[russian]{babel}
\usepackage{minted}

\usepackage{color} %% это для отображения цвета в коде
\usepackage{listings} %% собственно, это и есть пакет listings

\lstset{ %
language=C++,                 % выбор языка для подсветки (здесь это С++)
basic style=\small\sffamily, % размер и начертание шрифта для подсветки кода
numbers=left,               % где поставить нумерацию строк (слева\справа)
%numberstyle=\tiny,           % размер шрифта для номеров строк
step number=1,                   % размер шага между двумя номерами строк
numbersep=5pt,                % как далеко отстоят номера строк от подсвечиваемого кода
background color=\color{white}, % цвет фона подсветки - используем \usepackage{color}
show spaces=false,            % показывать или нет пробелы специальными отступами
showstringspaces=false,      % показывать или нет пробелы в строках
show tabs=false,             % показывать или нет табуляцию в строках
frame=single,              % рисовать рамку вокруг кода
tab size=2,                 % размер табуляции по умолчанию равен 2 пробелам
captions=t,              % позиция заголовка вверху [t] или внизу [b]
break lines=true,           % автоматически переносить строки (да\нет)
breakatwhitespace=false, % переносить строки только если есть пробел
escape inside={\%*}{*)}   % если нужно добавить комментарии в коде
}


\begin{document}
\begin{titlepage}
\begin{center}
\textbf{МИНИСТЕРСТВО ОБРАЗОВАНИЯ И НАУКИ РОССИЙСКОЙ ФЕДЕРАЦИИ
\medskip
МОСКОВСКИЙ АВИАЦИОННЫЙ ИНСТИТУТ
(НАЦИОНАЛЬНЫЙ ИССЛЕДОВАТЕЛЬСКИЙ УНИВЕРСТИТЕТ)
\vfill\vfill
{\Huge ЛАБОРАТОРНАЯ РАБОТА №2} \\
по курсу объектно-ориентированное программирование
I семестр, 2021/22 уч. год}
\end{center}
\vfill

Студент \uline{\it {Абаровский Олег Александрович, группа М8О-207Б-20}\hfill}

Преподаватель \uline{\it {Дорохов Евгений Павлович}\hfill}

\vfill
\end{titlepage}

\subsection*{Условие}

Разработать программу на языке C++ согласно варианту задания. Программа на C++ должна собираться с помощью системы сборки CMake. Программа должна получать данные из стандартного ввода и выводить данные в стандартный вывод. \newline
Необходимо настроить сборку лабораторной работы с помощью CMake. Собранная
программа должна называться oop\_exercise\_01 \newline
Задание: \newline
Вариант 1: Комплексное число в алгебраической форме представляются парой
действительных чисел (a, b), где a – действительная часть, b – мнимая
часть. Реализовать класс Complex для работы с комплексными числами.\newline
Обязательно должны быть присутствовать операции\newline
- сложения add, (a, b) + (c, d) = (a + c, b + d);\newline
- вычитания sub, (a, b) – (c, d) = (a – c, b – d);\newline
- умножения mul, (a, b) ´ (c, d) = (ac – bd, ad + bc);\newline
- деления div, (a, b) / (c, d) = (ac + bd, bc – ad) / (c 2 + d 2 );\newline
- сравнение equ, (a, b) = (c, d), если (a = c) и (b = d);\newline
- сопряженное число conj, conj(a, b) = (a, –b).\newline

Операции над объектами реализовать в виде перегрузки операторов.\newline
Реализовать пользовательский литерал для работы с константами объектов
созданного класса.

\subsection*{Описание программы}

Исходный код лежит в файле main.cpp, в котором реализован класс комплексного числа и описаны методы этого класса, среди которых сложение, вычитание, умножение, деление, сравнение, нахождение сопряженного числа, реализованные в виде перегрузки операторов.

\subsection*{Дневник отладки}
Исправлений не потребовалось.


\subsection*{Недочёты}


\subsection*{Выводы}
В ходе выполнения лабораторной работы я изучил такие понятия как перегрузка операторов и пользовательский литерал в c++ и создал класс с набором методов, реализованных в виде перегрузки операторов. Возможность применения стандартных понятных операторов к собственным классам показалась мне интересной и полезной для облегчения работы с программой.


\vfill

\subsection*{Исходный код}

{\Huge main.cpp}
\begin{minted}{c++}
#include <iostream> 
 
class Complex 
{
public:
  double re, im;
    Complex(){}
    Complex(double r, double i) 
    {
      re = r; im = i;
    }
    Complex& operator = (Complex);
    Complex operator + (Complex);
    Complex operator - (Complex);
    Complex operator * (Complex&);
    Complex operator / (Complex&);
    bool operator == (Complex& com); 
    Complex conj()
  {
      return Complex(re, -im);
  }
  ~Complex(){} 
};

Complex& Complex::operator = (Complex com) 
{  
    this->re = com.re; 
    this->im = com.im; 
    return *this;
}
 
bool Complex::operator==(Complex& com) 
{
    if(this->re == com.re && this->im == com.im) 
        return 1; 
    else 
        return 0;
}
 
Complex Complex::operator*(Complex &com) 
{  
    double i, j;  
    i = re * com.re - im * com.im; 
    j = re * com.im + com.re * im; 
    re = i; 
    im = j; 
    return *this; 
} 
 
Complex Complex::operator/(Complex &com) 
{  
    double i, j, k; 
    k = re * re +com.im * com.im; 
    i = (re * com.re + im * com.im) / k; 
    j = (com.re * im - re * com.im) / k; 
    re = i; 
    im = j; 
    return *this;
}
 
Complex Complex::operator+(Complex com) 
{ 
    this->re = this->re + com.re; 
    this->im = this->im + com.im; 
    return *this;
} 
 
Complex Complex::operator-(Complex com) 
{ 
    this->re = this->re - com.re; 
    this->im = this->im - com.im; 
    return *this;
}

double operator "" _R(long double re)
{
    return re;
}

double operator "" _I(long double im)
{
    return im;
}
 
int main() 
{
    double re, im;
  bool f;
  Complex com1, com2, com, comsub, Ccom1, Ccom2;
  printf("Enter complex number one\n");
  scanf("%lf %lf", &re, &im);
  com1 = Complex(re, im);
  printf("Enter complex number two\n");
  scanf("%lf %lf", &re, &im);
  com2 = Complex(re, im);
  Ccom1 = com1;
  Ccom2 = com2;
  printf("com1 = (%lf, %lf)\n", com1.re, com1.im);
  printf("com2 = (%lf, %lf)\n", com2.re, com2.im);
  com = com1 + com2;
  printf("Sum = (%lf, %lf)\n", com.re, com.im);
  com1 = Ccom1;
  com2 = Ccom2;
  comsub = com1 - com2;
  printf("Sub = (%lf, %lf)\n", comsub.re, comsub.im);
  com1 = Ccom1;
  com2 = Ccom2;
  com = com1 * com2;
  printf("Mul = (%lf, %lf)\n", com.re, com.im);
  com1 = Ccom1;
  com2 = Ccom2;
  com = com1 / com2;
  printf("Div = (%lf, %lf)\n", com.re, com.im);
  com1 = Ccom1;
  com2 = Ccom2;
  f = (com1 == com2);
  printf("Equ = %d\n", f);
  com1 = Ccom1;
  com2 = Ccom2;
  com1 = com1.conj();
  printf("Conj1 = (%lf, %lf)\n", com1.re, com1.im);
  com1 = Ccom1;
  com2 = Ccom2;
  com2 = com2.conj();
  printf("Conj2 = (%lf, %lf)\n", com2.re, com2.im);
}
\end{minted}
    \pagebreak


\end{document}
